\chapter{Discussions}
\label{cp:ds}
In modern literature, the Kronecker-Weber theorem is usually deduced as a simple consequence of class field theory. In fact, one can first investigate the theorem in local scenario where $K_\mathfrak{p}$ and $\mathbb{Q}_p$ is concerned. Then global-local principle can be utilized to extend it into global situation as we presented here. Obviously class field theory offers a more elegant path towards our goal.\\\\
However, by the time this paper is written, mathlib still lacks of support of local fields and relevant concepts, making it much more difficult to start from this angle.\\\\
Through out the paper, I've listed all relevant lemmas that is already available in mathlib, and explained the difference between them and their natural language version as commonly known. One can observe easily that many things are still missing, and everything in mathlib is written in the most general form possible - more often than not, it takes complicated process to downgrade them to the version we normally encounter. This incompleteness and complexity repels many from getting into formalization with Lean. Fortunately the community is working hard to smoothen the learning curve and make formalization more accessible, an example would be the recent work in Beijing International Center for Mathematical Research, where a project trying to connect mathlib with the Stacks project is on going.\\\\
Experienced readers might find section 3.1 tautology, and some might consider viewing proofs as algorithm weird, but this level of clarity is essential during formalization, and the algorithm viewpoint greatly enhanced computablity of the theorem.\\\\
There are indeed still much to do to completely formalize Kronecker-Weber theorem. For instance, higher ramification group is still on TO-DO list in mathlib. Also, proving the halting condition of our algorithm would be challenging, let alone all the details in last two proofs. But hopefully this paper can serve as a start heading to the target.