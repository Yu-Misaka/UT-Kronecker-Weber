\documentclass{standalone}
\usepackage{tikz, float, ulem, amsthm, tikz-cd}
\usepackage{quiver}

% Fonts
% \usepackage[utf8]{inputenc} % Required for international characters
\usepackage[T1]{fontenc} % Font encoding for international characters
% \usepackage{newpxtext} % Alternative use of the PX fonts
\usepackage{newpxmath} % Alternative use of the PX fonts (Math)
% \usepackage{microtype} % Slightly tweak font spacing for aesthetics
\usepackage{fvextra} % Compatibility with csquotes (?)
\usepackage{csquotes} % Compatibility with babel (?)
\linespread{1.2} % Increase line spacing slightly

% Ensure compilation with XeLaTeX or LuaLaTeX for fontspec
\usepackage{fontspec}
\setmainfont{TeX Gyre Pagella}[LetterSpace=0.7]

% Define a command to switch to Lato font
\newcommand{\latofont}{\fontspec{Lato}}
\usepackage{quiver}

\begin{document}

\begin{tikzcd}[column sep=tiny]
	{\rm{\textbf{INPUT}}} \\
	{\rm{has \ tame \ ramification?}} & {\rm{\textbf{OUTPUT}}} \\
	{\rm{take \ a \ tamely \ ramified \ prime}} \\
	{\rm{a \ field \ in \ which \ it \ is \ unramified}}
	\arrow[from=1-1, to=2-1]
	\arrow["{\rm{No}}", from=2-1, to=2-2]
	\arrow["{\rm{Yes}}", from=2-1, to=3-1]
	\arrow["{\rm{Theorem \ 3.1}}", from=3-1, to=4-1]
	\arrow[shift left=5, from=4-1, to=2-1]
\end{tikzcd}

\end{document}